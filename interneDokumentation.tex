% !TEX root = Projektdokumentation.tex
\section{Benutzerhandbuch}
\label{sec:api_documentation}
\vspace*{36 pt}
\vspace{4cm}
\begin{center}\Huge\bfseries\itshape
BiLang-API: \\
Technische Dokumentation
\end{center}	

\vspace{11cm}
\begin{center}\large
	Version 1.0.0 vom 19.05.2021 \\
	Autor: \companyName
\end{center}
\newpage


Dies ist eine technische Dokumentation für die BiLang-Übersetzungs-API. 
Sie ist die Schnittstelle für die beiden anderen Komponenten der BiLang-Software. 
Sowohl das JTL-Shop-Plugin als auch die Zusatzsoftware für die JTL-Wawi sind auf diese API angewiesen. 

Eine Anfrage an die API besteht grundsätzlich immer folgenden 4 Parametern:

\begin{tabular}{ |l|l| }
	\hline
	Parameter & Beschreibung \\
	\hline
	content & Zu Übersetzender Text. \\
	\hline
	source\_language\_iso & Ausgangssprache des Parameters 'content'. \\
	 & \(\cdot\) 'CS' - Tschechisch \\
	 & \(\cdot\) 'DE' - Deutsch \\
	 & \(\cdot\) 'EL' - Griechisch \\
	 & \(\cdot\) 'EN' - Englisch \\
	 & \(\cdot\) 'ES' - Spanisch \\
	 & \(\cdot\) 'FR' - Französisch \\
	 & \(\cdot\) 'HU' - Ungarisch \\
	 & \(\cdot\) 'IT' - Italienisch \\
	 & \(\cdot\) 'NL' - Niederländisch \\
	 & \(\cdot\) 'PL' - Polnisch \\
	 & \(\cdot\) 'PT' - Portugiesisch \\
	 & \(\cdot\) 'RU' - Russisch \\
	 & \(\cdot\) 'SV' - Schwedisch \\
	 \hline
	destination\_language\_iso & Zielsprache in die 'content' übersetzt werden soll. \\
	 & \(\cdot\) 'CS' - Tschechisch \\
	 & \(\cdot\) 'DE' - Deutsch \\
	 & \(\cdot\) 'EL' - Griechisch \\
	 & \(\cdot\) 'EN' - Englisch \\
	 & \(\cdot\) 'ES' - Spanisch \\
	 & \(\cdot\) 'FR' - Französisch \\
	 & \(\cdot\) 'HU' - Ungarisch \\
	 & \(\cdot\) 'IT' - Italienisch \\
	 & \(\cdot\) 'NL' - Niederländisch \\
	 & \(\cdot\) 'PL' - Polnisch \\
	 & \(\cdot\) 'PT' - Portugiesisch \\
	 & \(\cdot\) 'RU' - Russisch \\
	 & \(\cdot\) 'SV' - Schwedisch \\
	 \hline
	api\_key & Bei Vertragsschluss wird Ihnen ein API-Key übergeben. \\
	 & Bei Fragen wenden Sie sich bitte an den {\href{mailto:support@ris-development.de}{Support}}. \\
	\hline
\end{tabular}


Dabei kann 'content' auch Inhalte mit XML-Format beinhalten, die XML-Tags werden dann nicht mit übersetzt, 
um HTML-Strukturen wie zum Beispiel Attribute oder Anker korrekt beizubehalten.

\paragraph{Beispiele}

HTTP-Methode: POST auf bilang.ris-development.net/translation\_request
\begin{lstlisting}[extendedchars=\true,]
	{
		"content": "bÜersetzung", 
		"source_language_iso": "DE", 
		"destination_language_iso": "EN",
		"api_key": "9876543210"
	}
\end{lstlisting}


Response:
\begin{lstlisting}[extendedchars=\true,]
	Translation
\end{lstlisting}

 \vspace{2cm}

HTTP-Methode: POST auf bilang.ris-development.net/translation\_request
\begin{lstlisting}[extendedchars=\true,]
	{
		"content": "<a href="/büersetzung"> Übersetzung </a>",
		"source_language_iso": "DE",
		"destination_language_iso": "EN",
		"api_key": "9876543210"
	}
\end{lstlisting}

Response:
\begin{lstlisting}[extendedchars=\true,]
	<a href="/büersetzung"> Translation </a>
\end{lstlisting}

\paragraph{Limit}
Die maximale Anfragengröße darf 30kbytes nicht überschreiten, 

\paragraph{Formate und Sicherheit}
Die Parameter werden benutzt, um Informationen an die API zu übergeben. 
Dabei wird vorausgesetzt, dass alle Parameter den UTF8-Zeichensatz benutzen.
Die Antworten sind ebenfalls im UTF8-Zeichensatz.
Ihr API-Key ist absolut einmalig und wird bei jeder Anfrage überprüft.
Der gesamte Datenaustausch wird mit SSL abgesichert.