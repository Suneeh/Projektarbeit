\section{Schluss}
\subsection{Soll-Ist-Vergleich}
\begin{center}
    \begin{tabular}{ |l|r|r|r| }
        \hline
        \textbf{Projektphasen} & \textbf{Soll-Zeit [h]} & \textbf{Ist-Zeit [h]} & \textbf{Differenz [h]} \\
        \hline  
        \textbf{Projektplanung} & & & \\
        \hspace{0.65cm} Ist-Analyse & 2 & 2 & 0 \\
        \hspace{0.65cm} Soll-Konzept & 2 & 2 & 0 \\
        \hspace{0.65cm} Lösungsansätze vergleichen & 2 & 2 & 0 \\
        \hline  
        \textbf{Entwurf} & & & \\
        \hspace{0.65cm} Datenbankmodell Schnittstelle & 1 & 1 & 0\\
        \hspace{0.65cm} Datenbankmodell Zusatzsoftware & 2 & 2 & 0 \\
        \hspace{0.65cm} Aufbau Schnittstelle & 4 & 3 & \textcolor{Green}{-1}\\
        \hspace{0.65cm} Aufbau Zusatzsoftware & 4 & 6 & \textcolor{red}{+2} \\
        \hline  
        \textbf{Implementierung} & & & \\
        \hspace{0.65cm} \textbf{Schnittstelle} & & & \\
        \hspace{1.3cm} Erstellen der Datenbank & 5 & 4 & \textcolor{Green}{-1} \\
        \hspace{1.3cm} Logik zur Verarbeitung & 10 & 9 & \textcolor{Green}{-1} \\
        \hspace{1.3cm} Ausgabe der Metriken & 3 & 2 & \textcolor{Green}{-1} \\
        \hline  
        \hspace{0.65cm} \textbf{Zusatzsoftware} & & & \\
        \hspace{1.3cm} Datenbankintegration & 4 & 4 & 0\\
        \hspace{1.3cm} Erfassung von übersetzten Inhalten & 2 & 3 & \textcolor{red}{+1} \\
        \hspace{1.3cm} Im- und Exportformate & 5 & 6 & \textcolor{red}{+1}\\
        \hspace{1.3cm} Verarbeitung der Daten & 5 & 6 & \textcolor{red}{+1}\\
        \hline  
        \textbf{Testen} & & & \\
        \hspace{0.65cm} Testen der Schnittstelle & 3 & 3 & 0\\
        \hspace{0.65cm} Testen der Zusatzsoftware & 5 & 4 & \textcolor{Green}{-1} \\
        \hline  
        \textbf{Projektabschluss} & & & \\
        \hspace{0.65cm} Soll-Ist-Vergleich & 1 & 1 & 0\\
        \hspace{0.65cm} Dokumentation & 7 & 7 & 0\\
        \hspace{0.65cm} Abnahme & 1 & 1 & 0\\
        \hspace{0.65cm} Wirtschaftlichkeitsanalyse & 1 & 1 & 0\\
        \hline
        Gesamtstunden: & 70 & 70 & 0\\
        \hline
    \end{tabular}
\end{center}
Der Aufbau der Schnittstelle konnte durch die Verwendung von \gls{g:laravel} deutlich schneller als erwartet festgelegt werden. 
Ebenso wurde durch die vorgegebene Struktur von \gls{g:laravel} viel Zeit bei der Erstellung von Datenbanken und Klassen eingespart.

Dafür war die Handhabung der JTL-Wawi Datenbankzugriffe ein größeres Problem als ursprünglich geschätzt. 
Auch die Implementierung dauerte dadurch etwas länger als zunächst erwartet. 

\subsection{Dokumentation und Abnahme}
Es wurde eine \gls{g:api}-Dokumentation (siehe \ref{sec:api_documentation}.) für Endnutzer, sowie eine Installationsanleitung zur Zusatzsoftware (siehe \ref{sec:tool_documentation}.) erstellt. 
Die Software wurde anschließend zum Code-Review an ein Entwicklerteam abgegeben.

\subsection{Wirtschaftlichkeitsanalyse}
\subsubsection{Kosten}
Die Entwicklung des Projekts beläuft sich auf 70h. 
Hinzu kommen 70h die für die Entwicklung eines JTL-Shop-Plugins von einem Kollegen, wie in \ref{sec:soll} beschrieben.

Aus buchhalterischer Sicht fallen für einen Auszubildenden pro Stunde etwa 20€ brutto an Kosten an. Außerdem ist noch ein Gemeinkostenzuschlagsatz von etwa 10\% zu verrechnen. 
Somit betragen die gesamten Entwicklungskosten \( 3080\text{€} = ((70\text{h} + 70\text{h}) \cdot 20\text{€} ) + 10\%   \).

Die Einrichtung pro Kunde nimmt etwa 4h in Anspruch und muss ebenfalls mit Kosten von 20€/h berücksichtigt werden.

Die Kosten für die Übersetzung eines Shops ist abhängig von der Anzahl der zu übersetzenden Zeichen. 
Des weiteren sind die laufenden Kosten für Änderungen und neue Texte nur schwer einzuschätzen und stark abhängig von der Schnelllebigkeit des Sortiments. 

DeepL berechnet für 1.000.000 Zeichen exakt 20€ und eine monatliche Grundgebühr von 5€. 
Das bedeutet, dass pro Shop und pro Sprache initial mit etwa 60€ Kosten zu rechnen sind, und monatlich etwa 3€ anfallen.

\footnotetext[1]{Annahme: Ein Shop enthält 3 Mio. Zeichen sowie eine monatliche Erneuerungsrate von 5\%.}
\footnotetext[2]{Wird nur einmalig fällig. Bei weiteren Kunden muss der Betrag nicht erneut bezahlt werden.}

\paragraph{Beispiel für einen Kunden:}
\begin{center}
    \begin{tabular}{l r}
        \textbf{Kosten einmalig} & \textbf{Betrag} \\
        % 1. Entwicklung & 3080€ \\
        1. Einrichtung & 80€ \\
        2. initiale Übersetzung\footnotemark[1] & 60€ \\
        \hline
        Gesamt: & 140€ \\
    \end{tabular}
    \hspace{2.9cm}
    \begin{tabular}{l r}
        \textbf{Kosten monatlich} & \textbf{Betrag} \\
        % 1. Server/Domain & 5€ \\
        1. Bereitstellungspauschale\footnotemark[2] & 5€\\
        2. monatl. Zeichenmenge\footnotemark[1] & 3€\\
        \hline
        Gesamt: & 8€ \\
    \end{tabular}
\end{center}

\subsubsection{Einnahmen}
Das Abrechnungskonzept von DeepL bleibt gegenüber dem Kunden unverändert und besteht aus einer Bereitstellungspauschale, 
sowie eines Tarifs für die Anzahl der übersetzten Zeichen.
Die Bereitstellungspauschale soll pro Kunde im Jahr etwa 10\% der Entwicklungskosten betragen.
\[
    \frac{3080\text{€} \cdot 0.1}{12} = 25.66\text{€} \approx 25\text{€} / \text{Monat}
\]

Hinzukommend wird eine Einrichtungspauschale von 400€ einmalig fällig.
Außerdem werden die Übersetzungskosten mit einer Gewinnmarge von 100\% an den Kunden weitergegeben, daher verdoppeln sich alle Übersetzungskosten.

\begin{center}
        \begin{tabular}{ l r }
            \textbf{Kosten einmalig} & \textbf{Betrag} \\
            1. Einrichtungspauschale & 400€ \\
            2. initiale Übersetzung\footnotemark[1] & 120€ \\
            \hline
            Gesamt: & 520€ \\
        \end{tabular}
        \hspace{2.9cm}
        \begin{tabular}{l r}
            \textbf{Kosten monatlich} & \textbf{Betrag} \\
            1. Bereitstellungspauschale & 25€ \\
            2. monatl. Zeichenmenge\footnotemark[1] & 6€\\
            \hline
            Gesamt: & 31€ \\
        \end{tabular}
\end{center}
\footnotetext[1]{Annahme: Ein Shop enthält 3 Mio. Zeichen sowie eine monatliche Erneuerungsrate von 5\%.}


\subsubsection{Fazit}
Bei der Einrichtung ergibt sich somit ein Gewinn von 380€, sowie ein monatlicher Gewinn von etwa 18€ pro Kunde.

Somit kann berechnet werden, wie viel anschließend noch bis zu Amortisierung fehlt:
\[
    3080\text{€} - (3 \cdot 380\text{€}) = 1940\text{€}
\]

Berechnung der Amortisierung mit 3 Kunden anhand der laufenden Kosten:
\[
    \frac{1940\text{€}}{3 \cdot 18\text{€}} = 35.93 \text{ Monate} \approx 36 \text{ Monate}
\]

Mit drei Kunden und durchschnittlichen Nutzungswerten würde sich das Projekt nach 3 Jahren amortisieren. 
Das ist eine sehr weitsichtige Investition und damit sehr risikoreich.
Nach sorgfältiger Rücksprache mit den drei auftraggebenden Kunden wurde beschlossen, die Entwicklungskosten zu gleichen Teilen auf sie aufzuteilen.
Somit ist das Projekt von der Fertigstellung an amortisiert. Da die laufenden Kosten deutlich unter den laufenden Einnahmen liegen, 
können die Gewinne in die Instandhaltung, Wartung und Erweiterung der Software reinvestiert werden.
 

\subsection{Ausblick}
Das Projekt war erfolgreich und wird vorraussichtlich mehr als nur die auftraggebenden Kunden erreichen.

Es bieten sich noch einige Erweiterungsmöglichkeiten an, welche nicht mehr im Rahmen des Projekts abwickelbar waren. 
Beispielsweise eine GUI für das Auslesen und Beschreiben der \gls{g:api}-Datenbanken, um leichter Kunden hinzuzufügen oder deren Nutzungsdaten auszulesen.
Auch eine Automatisierung der monatlichen Abrechnung wäre denkbar und sinnvoll.

Je mehr Kunden das Produkt erhalten, desto wichtiger wird die Robustheit der \gls{g:api}, so könnte es in Zukunft relevant sein, ein mögliches \gls{g:bottleneck} in der \gls{g:api} zu verhindern.