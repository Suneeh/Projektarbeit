\section{Einleitung}
\subsection{Projektumfeld}
Meine Ausbildung zum Fachinformatiker für Anwendungsentwicklung absolviere ich in der Firma RIS Web- \& Software-Development GmbH \& Co. KG in Regensburg. 
Der Betrieb beschäftigt derzeit elf Mitarbeiter. 
Als Servicepartner von JTL-Software werden die Warenwirtschaftssoftware (JTL-Wawi) und der E-Commerce-Shop (JTL-Shop) betreut und mit Plugins oder anderen kleinen Softwareerweiterungen zur Abdeckung bestimmter Anforderungen erweitert. 
Für den JTL-Shop werden auch eigene Frontend-Templates umgesetzt. Für die JTL-Wawi zählen Einrichtung, Betreuungen und Schulungen zu den Hauptbereichen. 
Neben der Tätigkeit als Servicepartner werden auch Individuallösungen für Kunden entwickelt und betreut. 
Dazu zählen Entwicklungen von Corporate Designs, Social-Media-Marketing, Neugestaltung/Betreuung von Webseiten, Erstellung von Web-Applikationen und nativer Software.


\subsection{Ist-Analyse}
% Antrag
Unserer Kunden nutzen die E-Commerce-Software JTL-Shop zusammen mit dem Warenwirtschaftsprogramm JTL-Wawi.
Diese werden standardmäßig auf Deutsch installiert, mit der anschließenden Möglichkeit mehrere Sprachen zu nutzen. 
Bei unseren Kunden besteht eine rege Nachfrage nach Übersetzungen in andere Sprachen. 
Derzeit werden diese oft vom Kunden händisch durchgeführt, was bei vielen Artikeln zu sehr viel Arbeitsaufwand führt.
Außerdem müssen bei einer Änderung eines bereits bestehenden Inhalts auch die anderssprachigen Gegenstücke neu übersetzt werden, um Inkonsistenzen zu vermeiden.
%
Im JTL-Ökosystem gibt es zwei für uns relevante Teilbereiche. Zum einen den JTL-Shop, welcher als \gls{g:php}-Web-Applikation betrieben wird und hauptsächlich Fließtexte, aber auch kurze Satzteile in einer Datenbank speichert.
Zum anderen die JTL-Wawi, ein Warenwirtschaftsprogramm, welches auf einem Windows-System agiert, in dem die Artikel, Kategorien sowie Attribute, Beschreibungen und \gls{g:seo}-Inhalte erstellt, und in einer eigenen \gls{g:mssql}-Datenbank gespeichert werden.
Die JTL-Wawi lädt diese Texte außerdem über einen sogenannten Onlineshop-Abgleich in den Shop hoch.
Für beide Softwaresysteme müssen die, gegebenenfalls genutzten, HTML-Tags berücksichtigt werden.

\subsection{Soll-Zustand}
\label{sec:soll}
Zu Entwickeln sind daher ein JTL-Shop Plugin, eine Zusatzsoftware zur JTL-Wawi sowie eine Schnittstelle für uns. 
Diese Schnittstelle wird die Kommunikation mit dem tatsächlichen Übersetzungs-Service übernehmen.
Dabei soll sie zu übersetzende Texte entgegennehmen, an einen Übersetzungs-Dienst übertragen und die übersetzten Inhalte zurück liefern, sowie Nutzungsmetriken pro Kunde erfassen, zur Erhebung von Übersetzungsgebühren.

Meine Aufgaben werden die Entwicklung und Konzeption der Schnittstellen-Software und die Umsetzung der Zusatzsoftware auf dem Windowssystem sein.
Damit soll die Übersetzung automatisiert an einen Drittanbieter ausgelagert werden, um den Arbeitsaufwand und mögliche Fehler zu minimieren.

\subsection{Lösungsansätze}
\label{sec:solutions}
Bei Automatisierungen in einem großen Ausmaß, ist es wichtig den richtigen Ausführungszeitpunkt zu wählen. 
Dabei gilt es verschiedene Möglichkeiten abzuwägen, beispielsweise feste Uhrzeiten, oder stündliche Intervalle sowie das Aufrufen, wenn es einen neuen übersetzbaren Wert gibt. 
Eine feste Uhrzeit bei all unseren Kunden würde zu extremer Belastung der Schnittstelle führen, welche als API umgesetzt wird.
Mehrmals täglich zu übersetzen, bietet den großen Vorteil, dass Änderungen schneller übernommen werden können und die Auslastung der \gls{g:api} würde sich besser verteilen. 
Zusätzlich wäre ein Option eines manuellen Anstoßes sicherlich von Vorteil.

Eine weitere Unklarheit, die vor der Umsetzung der Zusatzsoftware genauer betrachtet werden sollte, ist die Art der Datenbankzugriffe auf die Datenbank der JTL-Wawi.
Es gibt zwei Arten auf die \gls{g:mssql}-Datenbank zuzugreifen, direkt über \gls{g:sql} oder über die von JTL bereitgestellte Software JTL-Ameise. 
Sie ermöglicht einen überwachten und einwandfreien lesenden und schreibenden Zugriff auf die Datenbank. 
Dabei können große Mengen an Anpassungen, der gleichen Art, in einem Durchlauf durchgeführt werden, was für die Übersetzungen optimal ist.
Diese Software ist jedoch erfahrungsgemäß sehr langsam (ein Export mit etwa 50.000 Artikeln braucht je nach Server etwa 4 Stunden).
Der direkte Zugriff auf die Datenbank mit Hilfe von \gls{g:sql} ist deutlich schneller, 
jedoch sprechen die stetigen Datenbank-Schema-Veränderungen bei Updates, sowie fehlende \gls{g:storedprocedure}s dagegen.
Kleinste Fehler können zu Inkonsistenzen in den Tabellen oder Datenverlust führen.
Da die Datenbank mit sehr sensiblen Kundendaten arbeitet und die Zugriffe nicht für jede Wawi-Version exakt gleich sind, fällt die Wahl auf JTL-A

Die Auswahl der Technologien und Systeme für die (REST-)\gls{g:api} ist groß, daher es gibt hier vor allem Entscheidungskriterien, der Wartbarkeit und Erweiterbarkeit. 
In meinem Betrieb ist \gls{g:php} allgegenwärtig und daher im Vergleich zu JavaScript deutlich wartbarer für alle Kollegen. 
Die Entscheidung, die \gls{g:api} in \gls{g:php} zu programmieren, fiel daher recht schnell und einfach.

Für die Übersetzungssoftware wurden verschiedene Anbieter wie Amazon Translate, Google Translate, Microsoft Translator und DeepL Translator getestet. 
Dabei konnten die namenhaften Dienstleister leider entweder qualitativ nicht mithalten, oder bieten keine \gls{g:api}, die aufgerufen werden kann. 
Daher soll der Übersetzungsdienst von DeepL benutzt werden.
Er bietet nicht nur sehr gute Übersetzungen, sondern außerdem auch eine Vielzahl an verschiedenen Ein-/ und Ausgangssprachen sowie die Option, HTML zu berücksichtigen. 
Die Nutzung des Services via \gls{g:api} ist zwar nicht kostenlos, jedoch wird für die sehr gute Übersetzungsqualität nur ein kleiner Betrag pro Zeichen fällig.


\subsection{Technologieplattform}

Die Nutzung eines \gls{g:framework} wie \gls{g:laravel} bietet sich bei der Arbeit mit \gls{g:php} zusätzlich an, da die Entwicklung vereinfacht und beschleunigt werden kann. 
Dank der hervorragenden Dokumentation hat sich \gls{g:laravel} in der Vergangenheit als sicher, schnell, zuverlässig und auf Grund des Model-View-Controller-Entwurfsmusters (\gls{g:mvc}) auch als sehr übersichtlich herausgestellt. 
Dabei repräsentieren \emph{Models} Daten, die von einem \emph{Controller} verarbeitet werden und dann mit Hilfe einer \emph{View} als (optische) Ausgabe dargestellt werden können.
\gls{g:laravel} bietet außerdem standardmäßig eine MySQL-Unterstützung, welche für das Projekt genutzt werden wird.

Da die JTL-Wawi das Microsoft .NET Framework voraussetzt, kann auch eine C\#-Applikation problemlos ausgeführt werden. 
Daher bietet sich C\# als Programmiersprache an und wird im Projekt für die Umsetzung der Zusatzsoftware benutzt. 

Für die Datenbankzugriffe wird das von JTL bereitgestellte Programm JTL-Ameise verwendet, da es erlaubt alle Datenbanktabellen der JTL-Wawi zu ex-/ bzw. importieren. 
Das Programm arbeitet mit \gls{g:csv}-Dateien, was eine angenehme Dateiverarbeitung ermöglicht und mit C\# leicht verarbeitet werden kann.

Zuletzt gilt es den Dienst zu erwähnen, welcher für die Übersetzung genutzt wird. 
DeepL hat sich als eine sehr hochwertige Übersetzungssoftware entpuppt, daher wird sie in diesem Projekt ihre Anwendung finden.

Da bereits die DeepL-\gls{g:api} über das \gls{g:json}-Format kommuniziert, bietet sich dieses auch für die \gls{g:middleware} an.