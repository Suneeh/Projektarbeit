\section{Entwurfsphase}
\subsection{Schnittstelle}

\subsubsection{Datenbankmodell}
Da die \gls{g:api} Anfragen empfängt und diese den richtigen Kunden für die Abrechnung zuordnen soll, braucht es zwei Tabellen. 
Eine für den Kunden ('customers') und eine für dessen Übersetzungsanfragen ('translation\_requests').
In 'customers' wird der Kunde und sein \gls{g:api}-Schlüssel zur Identifikation gespeichert, 
während in 'translation\_requests' der Inhalt, dessen Zeichenlänge, ein Zeitstempel so wie die Verknüpfung zum Kunden hinterlegt werden soll.
Für die Abrechnung muss eine Übersetzungsanfrage immer genau einem Kunden zugewiesen werden können (siehe Anhang \ref{sec:erd-api}).

\subsubsection{Aufbau}
\gls{g:rest}-\gls{g:api} steht für \emph{Representational State Transfer Application Programming Interface}. 
Der Hauptanwendungsbereich dieser Schnittstellen sind Web-Applikationen, welche mittels \gls{g:http} Anfragen, wie \texttt{GET}, \texttt{PUT}, \texttt{POST} oder \texttt{DELETE}, Daten zwischen Softwaresystemen austauschen können.
Dabei werden in diesem Fall die Datenpakete im \gls{g:json}-Format übertragen, um eine standardisierte Strukturierung zu ermöglichen.

Eine \gls{g:api} hat oft mehrere Endpunkte, welche jeweils mit unterschiedlichen \gls{g:http}-Methoden genutzt werden können (siehe Anhang \ref{sec:table_http_request}).
Die \gls{g:api} wird mit dem \gls{g:php}-\gls{g:framework} \gls{g:laravel} erstellt. 
Dabei folgt es dem \gls{g:mvc}-Entwurfsmuster, welches sich durch die Trennung der Anwendungslogik von Darstellung und Benutzerinteraktionen auszeichnet.

Die \emph{Models} der \gls{g:api} beziehen sich auf die beiden Datenbanktabellen (siehe Anhang \ref{sec:table_tranlation_request} und \ref{sec:table_customers}), und können aus einer Tabellenzeile jeweils ein Objekt formen.
Der \emph{View} gibt übersetzte Inhalte in Form eines \gls{g:json}-Objekts zurück.

Außerdem wird ein \emph{Controller} für die Verarbeitung der \emph{Requests} benutzt.
Er wird Texte empfangen, eine Übersetzungsanfrage an DeepL stellen und anschließend den übersetzen Wert zurückgeben.
Die gesamte Verarbeitungslogik passiert ebenfalls im Controller.

\subsection{Zusatzsoftware}
\subsubsection{Datenbankmodell}
Um einen vorliegenden Datensatz nicht zu übersetzen, obwohl dieser schon einmal übersetzt wurde, muss abgespeichert sein, dass es bereits eine Übersetzung in die jeweilige Sprache gibt. 
Ansonsten entstehen Kosten und Zeitaufwand/Auslastung am Server, die man auch vermeiden kann.

Um mögliche Änderungen in einem Text erkennen zu können, muss ein Indikator entworfen werden, der schnell auslesbar ist und wenn möglich Versionen, anhand der Länge oder eines Zeitstempels erkennt.
Dieser wird anschließend, zusammen mit einem eindeutigen Identifikator des übersetzten Textes in einer Datenbanktabelle gespeichert (siehe Anhang \ref{sec:erd-csharp}).

\subsubsection{Aufbau}
Die Zusatzsoftware wird in C\# und damit auch objektorientiert programmiert. 
Sie wird die Texte, mit Hilfe der JTL-Ameise, aus der Datenbank der JTL-Wawi auslesen und anschließend diese gelesenen Werte an die \gls{g:api} schicken.
Die Antworten des Übersetzungsdienstes werden dann als neue Datensätze in eine, für die JTL-Ameise lesbare \gls{g:csv}-Datei, geschrieben und importiert.

Um die Verarbeitungslogik von den Im-/ Exporten sowie den \gls{g:api}-Aufrufen zu trennen, werden verschiedene Klassen entworfen.
Dabei werden die Kommunikationswege zwischen der Zusatzsoftware und der JTL-Ameise bzw. \gls{g:api} in jeweils eine eigene Klasse gespalten. 
Außerdem soll die Verarbeitung und Aufbereitung der \gls{g:csv}-Dateien ebenso vom restlichen Programmcode separiert werden. 
