\section{Testphase}
\subsection{Schnittstelle}
Die Datenbankanbindung wurde bereits während der Entwicklung getestet und mit Hilfe eines MySQL-Plugins direkt in der Entwicklungsumgebung überprüft.
Da die \gls{g:laravel}-Controller mit Hilfe der \emph{Models} bereits in der Lage sind, die Datenbank zu beschreiben, wenn ein passendes \emph{Model} vorliegt, mussten hier keine weiteren Tests gemacht werden.
Somit wurden die Datenbanktabellen lediglich auf ihre Richtigkeit sichtgeprüft. 

Die Schnittstelle dient als \gls{g:middleware} und gibt Fehler, die bei der Verbindung mit der DeepL-\gls{g:api} auftreten, direkt weiter an den Endnutzer.
Bei Fehlern in der Schnittstelle selbst werden Statuscodes in die Datenbank geschrieben.

Ansonsten wurden, mittels Testanfragen, vor allem überprüft, dass keine Routen für den Endnutzer verfügbar sind, mit denen Daten aus der Datenbank gelesen werden können.

Auch die Validierung in der \texttt{store()} Methode des \texttt{TranslationRequestController} wurde ausgiebig überprüft, indem verschiedene fehlerhafte Parameter an die Route übersendet wurden. 

\subsection{Zusatzsoftware}
Für die Zusatzsoftware zur JTL-Wawi wurden vor allem die einzelnen Klassen an sich getestet. 

Zuerst muss das Programm eine problemfreie Verbindung mit der Datenbank herstellen können, um auch schnell und zuverlässig eine Überprüfung durchführen zu können.

Anschließend wurde überprüft, ob der Aufruf der JTL-Ameise korrekt funktioniert. 
Das kann lediglich von einem Menschen kontrolliert werden, indem die CSV-Datei auf ihre Integrität sichtgeprüft wird.

Um die Anbindung an die \gls{g:api} zu testen, wurde im Controller ein \texttt{api\_key} hinterlegt, für den kein Übersetzungsprozess angestoßen wird, 
sondern lediglich das Kürzel der Zielsprache mit dem Ausgangstext in Klammern ausgegeben wird (siehe Anhang \ref{sec:testing_api}).
So können größere Tests durchgeführt werden, ohne Kosten zu verursachen, solange der Test-\gls{g:api}-Schlüssel in der Anfrage angegeben wird.