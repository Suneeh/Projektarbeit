% !TEX root = Projektdokumentation.tex

\newglossaryentry{g:dotnet}{
    name={.NET},
    description={Microsoft .NET; Überbegriff für verschiedene Software-Plattformen zur Ausführung und Entwicklung von Programmen}
}
\newglossaryentry{g:api}{
    name={API},
    description={Application Programming Interface; Eine Programmierschnittstelle zur von Software zu Software}
}
\newglossaryentry{g:bottleneck}{
    name={Bottleneck},
    description={ein Engpass in einem Prozess, in dem die Auslastung besonders hoch ist}
}
\newglossaryentry{g:csv}{
    name={CSV},
    description={Comma seperated values; Dateiformat mit dem meist Tabellen dargestellt werden. Einzelne Spalten werden mit einem Komma [,] getrennt}
}
\newglossaryentry{g:framework}{
    name={Framework},
    description={Bereitstellung eines komponentenbasierten Grundgerüsts}
}
\newglossaryentry{g:gui}{
    name={GUI},
    description={Graphical User Interface; Eine grafische Bedienoberfläche einer Software}
}
\newglossaryentry{g:hashwert}{
    name={Hashwert},
    description={ist eine Art Fingerabdruck eines Datensatzes}
}
\newglossaryentry{g:html}{
    name={HTML},
    description={Hypertext Markup Language; Auszeichnungssprache zur Gliederung von Inhalten. Bildet das Grundgerüst des WWW}
}
\newglossaryentry{g:http}{
    name={HTTP},
    description={Hyper Text Transfer Protokol; Überwiegend in Web-Anwendungen verwendetes zustandsloses Datenübertragungsprotokoll zur Weitergabe von Daten an den Anwender}
}
\newglossaryentry{g:json}{
    name={JSON},
    description={JavaScript Object Notation; Ist ein Programmiersprachenunabhangiges Datenformat, um Daten in einer festgelegten Struktur an den Anwender zu übergeben}
}
\newglossaryentry{g:laravel}{
    name={Laravel},
    description={Framework im Entwurfsmuster Model-View-Controller, dass Bausteine zur Entwicklung von Web-Applikationen bereitstellt}
}
\newglossaryentry{g:middleware}{
    name={Middleware},
    description={ist eine Software die zwischen zwei Anwendungen vermittelt}
}
\newglossaryentry{g:mssql}{
    name={MSSQL},
    description={Microsoft SQL Datenbank}
}
\newglossaryentry{g:mvc}{
    name={MVC},
    description={Model-View-Controller; Ist ein Entwurfsmuster zur Unterteilung der Software-Elemente in Model,View und Controller zur besseren Wiederverwendbarkeit der einzelnen Elemente}
}
\newglossaryentry{g:mysql}{
    name={MySQL},
    description={weitverbreitetes kostenloses Datenbanksystem zur Verwaltung relationaler Datenbanken}
}
\newglossaryentry{g:php}{
    name={PHP},
    description={Hypertext Preprocessing; Weit verbreitete Skript-Sprache, überwiegend in der Web-Entwicklung genutzt}
}
\newglossaryentry{g:rest}{
    name={REST},
    description={Representational State Transfer; Din Architekturstil für die Kommunikation zwischen zwei Softwaresystemen}
}
\newglossaryentry{g:seo}{
    name={SEO},
    description={Search Engine Optimization; Die Optimierung einer Webseite, zur besseren Sichtbarkeit für Suchmaschinen, wie zum Beispiel Google}
}
\newglossaryentry{g:sql}{
    name={SQL},
    description={Structured Query Language}
}
\newglossaryentry{g:storedprocedure}{
    name={Stored-Procedure},
    description={eigenständiger SQL-Befehl der für bestimmte Prozeduren gespeichert wurde}
}
\newglossaryentry{g:url}{
    name={URL},
    description={Uniform Resource Locator; Dient der eindeutigen Adressierung von Dateien und Verzeichnissen und wird hauptsächlich im Internet genutzt}
}
\newglossaryentry{g:win_cron}{
    name={Windows-Aufgabenplanung},
    description={ist eine Windowsfunktion, die es erlaubt Programme zu festgelegten Zeitpunkten zu starten}
}
\newglossaryentry{g:xml}{
    name={XML},
    description={Extensible Markup Language; Wird zur Darstellung und zum Austausch von hierarchisch strukturierterten Daten benutzt}
}